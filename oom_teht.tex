\documentclass[a4paper,12pt,titlepage]{article}
\usepackage{amssymb,amsthm,amsmath} %ams
\usepackage[finnish]{babel} %suomenkielinen tavutus
\usepackage[T1]{fontenc}    % kirjasimen kooditaulukko
\usepackage[utf8]{inputenc} % skandit utf-8 koodauksella (uudet tiedostot)
\usepackage{listings}
\usepackage{color}
%\usepackage[latin1]{inputenc} % skandit utf-8 koodauksella (vanhat tiedostot)

\usepackage{graphicx} %dokumentti sisältää eps-muotoisia kuvia
\usepackage{pstricks,pst-node,pst-text,pst-3d} %PostScript olioiden tekoon
%\usepackage{icomma} % desimaalipilkku kaavoissa

\linespread{1.24} %riviväli 1.5
\sloppy % Vähentää tavutuksen tarvetta, "leventämällä" rivin keskellä olevia välilyöntejä.


\usepackage{listings}
\usepackage{color}

\definecolor{dkgreen}{rgb}{0,0.6,0}
\definecolor{gray}{rgb}{0.5,0.5,0.5}
\definecolor{mauve}{rgb}{0.58,0,0.82}

\lstset{ % General setup for the package
	language=Java,
	basicstyle=\small\sffamily,
	numbers=left,
 	numberstyle=\tiny,
	frame=tb,
	tabsize=4,
	columns=fixed,
	showstringspaces=false,
	showtabs=false,
	keepspaces,
	commentstyle=\color{green},
	keywordstyle=\color{blue}
}

%%%%%%%%%%%%%%%%%%%%%%%%%%%%%%%%%%%%%%%%%%%%%%%%%%%%%%%%%%%%%%%%%%%%%%%%%%%%%%%%%%%%%%%%%%%%%%
%
% Lauseille, määritelmille ja muille vastaaville voidaan määritellä omat ympäristöt, jolloin
% niille saadaan yhtenäinen ulkoasu. Numerointi tapahtuu lukukohtaisesti ja useampi ympäristö
% linkitetään käyttämään samaa laskuria.
%
% Ympäristöt maar, lemma, lause, esim, huom käyttävät
% Ympäristölle 'lemma' voitaisiin määritellä myös oma laskurinsa.
% \newtheorem{lemma}{Lemma}[section]
%
%%%%%%%%%%%%%%%%%%%%%%%%%%%%%%%%%%%%%%%%%%%%%%%%%%%%%%%%%%%%%%%%%%%%%%%%%%%%%%%%%%%%%%%%%%%%%%

\theoremstyle{definition}
\newtheorem{maar}{Määritelmä}[section]
\newtheorem{lemma}[maar]{Lemma}
\newtheorem{seur}[maar]{Seuraus}
\newtheorem{lause}[maar]{Lause}
\newtheorem{esim}[maar]{Esimerkki}

% englanninkieliset mukaan mukavuussyistä
\newtheorem{definition}{Määritelmä}[section]
%\newtheorem{lemma}[maar]{Lemma}
\newtheorem{corollary}[maar]{Seuraus}
\newtheorem{theorem}[maar]{Lause}
\newtheorem{example}[maar]{Esimerkki}

\theoremstyle{remark}
\newtheorem{huom}[maar]{Huomautus}
\newtheorem*{huom*}{Huomautus}

% englanninkieliset mukaan mukavuussyistä
\newtheorem{remark}[maar]{Huomautus}

\makeatletter
\newenvironment{tod}[1][\todname]{\par
  \pushQED{\qed}%
  \normalfont \topsep6\p@\@plus6\p@\relax
  \trivlist
  \item[\hskip\labelsep
        \itshape
    #1\@addpunct{.}]\ignorespaces
}{%
  \popQED\endtrivlist\@endpefalse
}
\providecommand{\todname}{Todistus}
\def\textfontii{\the\textfont\tw@}
\def\AmSTeX{{\textfontii A\kern-.1667em%
  \lower.5ex\hbox{M}\kern-.125emS}-\TeX\spacefactor1000 }
\makeatother

% Yleisimmin käyttettäville komennoille voi määritellä lyhynnemerkintöjä
% esimerkiksi
\newcommand{\Q}{\mathbb{Q}}
\newcommand{\R}{\mathbb{R}}
\newcommand{\Z}{\mathbb{Z}}
\newcommand{\C}{\mathbb{C}}
\newcommand{\abs}[1]{\left\vert{#1}\right\vert} % Itseisarvo


%%%%%%%%%%%%%%%%%%%%%%%%%%%%%%%%%%%%%%%%%%%%%%%%%%%%%%%%%%%%%%%%%%%%%%%%%%%%%%%%
%
%   käyttäjän täytettävä alue alkaa
%
%%%%%%%%%%%%%%%%%%%%%%%%%%%%%%%%%%%%%%%%%%%%%%%%%%%%%%%%%%%%%%%%%%%%%%%%%%%%%%%%
% käyttäjän omat määrittelyt
\newcommand{\bx}{\mathbf{x}}

\title{Olio-ohjelmoinnin metodiikka demo1}
\author{Petri Holopainen, Matias Lappalainen, Topi Koivunen ja Oskari Vahala\\[1cm]
}
\date{Syyskuu 2018}

\begin{document}    % tekstin muodostus alkaa
\maketitle          % tuottaa nimiön
%\tableofcontents    % tuottaa sisällysluettelon
%\clearpage % poista kommentti merkki, jos haluat sivunvaihdon sisällysluettelon jälkeen


\section{Teht 1}
\title{c)}
\newline
Ei ole luokkaa, jolla olisi toiminnallisuutta värien lisäykseen, eli päivitysten tekeminen olisi hankalaa. Värejä on vain kymmenen, joten main-luokkaan täytyy lisätä värejä tarvittaessa. Sovelluskehitykseen ohjelma on todella huono ja puutteellinen. Värejä voi määritellä vain heksalla, ei rgba:lla tai muilla värinmääritysmenetelmillä.


\addcontentsline{toc}{section}{Kirjallisuutta} % lisätään sisällysluettelon tietoihin
\renewcommand{\refname}{Kirjallisuutta} % otsikon muutos

\newpage

\begin{lstlisting}
public class Colors {

    ArrayList<Color> Colors;

    public Colors() {
        String[] colors = {"#33a1ee", "#a958a5", "#cb3f68", "#2d406c", "#248e82", "#c481fb", "#28666e", "#f06261", "#ffffff", "#000000"};
        Colors = new ArrayList<Color>();


        for(String c : colors) {
            Colors.add(new Color(c, c));
        }
    }

    /**
     * Method will pick one pre defined colour from arrayList
     * @.pre (ArrayList =! null) &&  0<=tmp<10
     * @.post RESULT.length == 7
     */
    public String randomize(ArrayList<Color> Colors) {
        Random r = new Random();
        int tmp = r.nextInt(9);
        return Colors.get(tmp).getName();
    }


    public ArrayList<Color> getColors() {
        return Colors;
    }
}
\end{lstlisting}

\newpage

\begin{lstlisting}
class Color {

    // Define class parameters
    String name;
    String hex;

    public Color(String name, String hex) {
        this.hex = hex;
        this.name = name;
    }

    /**
     * Method compares two colors, and returns boolean,
     * if they are equal
     * @.pre != null
     * @.post RESULT === (comparable.length === 7 && comparable[0] === '#')
     */
    public boolean isEqual(Color comparable) {
        return this.hex.equals(comparable.getName());
    }

    // Return color name
    public String getName() {
        return this.name;
    }

    // Return color name
    public String getHex() {
        return this.hex;
    }

}

\end{lstlisting}

\newpage
\section{Teht 2}
\title{a)}

\begin{lstlisting}
/**
     * @.pre true
     * @.post sudoku.length == koko
     */
    public Sudoku() {
        this.sudoku = new int[koko][koko];
    }
\end{lstlisting}
\title{b)
\begin{lstlisting}
 protected boolean onSudoku() {
        for (int i = 0; i < 9; i++) {

            /*Tee uudet listat,
             joiden avulla voidaan vertailla
            */
            int[] row = new int[9];
            int[] square = new int[9];

            // cloonaa sudoku i rivi
            int[] column = sudoku[i].clone();

            /*
                Lisaa sudokun alkioita listoihin
            */
            for (int j = 0; j < 9; j ++) {
                row[j] = sudoku[j][i];

                // Kay lapi sudoku ruudukon
                square[j] = sudoku[(i / 3) * 3 + j / 3][i * 3 % 9 + j % 3];
            }

            // Jos jokin tarkistus on false niin palauta false
            if (!(tarkista(column) && tarkista(row) && tarkista(square)))
                return false;
        }
        return true;
    }
\end{lstlisting}}
\title{c)}
B-kohdan metodi onSudoku kutsuu tarkista-metodia
\begin{lstlisting}
    // Tarkistaa onko listassa kaikki 1-9 numerot
    protected boolean tarkista(int[] check) {
        int i = 0;
        Arrays.sort(check);
        for (int number : check) {
            if (number != ++i)
                return false;
        }
        return true;
    }
\end{lstlisting}
\title{d)}
\begin{lstlisting}
    public void rotateSudoku() {
        for (int i = 0; i < koko / 2; i++) {

            for (int j = i; j < koko-i-1; j++) {
                int temp = sudoku[i][j];

                sudoku[i][j] = sudoku[j][koko-1-i];

                sudoku[j][koko-1-i] = sudoku[koko-1-i][koko-1-j];

                sudoku[koko-1-i][koko-1-j] = sudoku[koko-1-j][i];

                sudoku[koko-1-j][i] = temp;
            }
        }
    }
\end{lstlisting}
\title{e)}
\begin{lstlisting}
import org.junit.jupiter.api.*;
import org.junit.platform.runner.JUnitPlatform;
import org.junit.runner.RunWith;

@DisplayName("Testing Sudoku")
class SudokuTest {

    Sudoku sudoku;

    @BeforeEach
    void init() {
        System.out.println("Running test");
    }

    @RepeatedTest(1000)
    void test1() {

        sudoku = new Sudoku();

        sudoku.generateSudoku();

        boolean case1 = sudoku.onSudoku();
        sudoku.rotateSudoku();
        boolean case2 = sudoku.onSudoku();

        Assertions.assertEquals(case1, case2);
    }

}

\end{lstlisting}
\newpage
\section{Teht 3}
\title{a)}
\newline
Tehtävän ensimmäinen metodi palauttaa väriolion annetuista koordinaateista.

\begin{lstlisting}
    /**
    ** @.pre |x| < width/2 && |y| < height/2 && x != null && y != null
    ** @.post RESULT == COLOR
    */
    readCell(int x, int y)
\end{lstlisting}
WriteCell metodi lisää väriolion koordinaatteihin x, y.
\begin{lstlisting}
    /**
    ** @.pre x != null && y != null && Color == Color
    ** @.post true
    */
    writeCell(int x, int y, Color)
\end{lstlisting}
Width ja height kertovat ruudukon dimensiot. Ainakin readCell metodin määrittelyä tulee muuttaa siten, että methodi voi palauttaa myös undefinedia, sillä ruudukon muistia ei ole alustettu. Määrittely ei ota kantaa algoritmiin, koska määrittelyä ei tarvitse muuttaa, sillä negatiivisuus otetaan huomioon alkuperäisessä.



\end{document}