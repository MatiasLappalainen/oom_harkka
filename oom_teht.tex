\documentclass[a4paper,12pt,titlepage]{article}
\usepackage{amssymb,amsthm,amsmath} %ams
\usepackage[finnish]{babel} %suomenkielinen tavutus
\usepackage[T1]{fontenc}    % kirjasimen kooditaulukko
\usepackage[utf8]{inputenc} % skandit utf-8 koodauksella (uudet tiedostot)
\usepackage{listings}
\usepackage{color}
%\usepackage[latin1]{inputenc} % skandit utf-8 koodauksella (vanhat tiedostot)

\usepackage{graphicx} %dokumentti sisältää eps-muotoisia kuvia
\usepackage{pstricks,pst-node,pst-text,pst-3d} %PostScript olioiden tekoon
%\usepackage{icomma} % desimaalipilkku kaavoissa

\linespread{1.24} %riviväli 1.5
\sloppy % Vähentää tavutuksen tarvetta, "leventämällä" rivin keskellä olevia välilyöntejä.


\usepackage{listings}
\usepackage{color}

\definecolor{dkgreen}{rgb}{0,0.6,0}
\definecolor{gray}{rgb}{0.5,0.5,0.5}
\definecolor{mauve}{rgb}{0.58,0,0.82}

\lstset{frame=tb,
  language=Java,
  aboveskip=3mm,
  belowskip=3mm,
  showstringspaces=false,
  columns=flexible,
  basicstyle={\small\ttfamily},
  numbers=none,
  numberstyle=\tiny\color{gray},
  keywordstyle=\color{blue},
  commentstyle=\color{dkgreen},
  stringstyle=\color{mauve},
  breaklines=true,
  breakatwhitespace=true,
  tabsize=3
}

%%%%%%%%%%%%%%%%%%%%%%%%%%%%%%%%%%%%%%%%%%%%%%%%%%%%%%%%%%%%%%%%%%%%%%%%%%%%%%%%%%%%%%%%%%%%%%
%
% Lauseille, määritelmille ja muille vastaaville voidaan määritellä omat ympäristöt, jolloin
% niille saadaan yhtenäinen ulkoasu. Numerointi tapahtuu lukukohtaisesti ja useampi ympäristö
% linkitetään käyttämään samaa laskuria.
%
% Ympäristöt maar, lemma, lause, esim, huom käyttävät
% Ympäristölle 'lemma' voitaisiin määritellä myös oma laskurinsa.
% \newtheorem{lemma}{Lemma}[section]
%
%%%%%%%%%%%%%%%%%%%%%%%%%%%%%%%%%%%%%%%%%%%%%%%%%%%%%%%%%%%%%%%%%%%%%%%%%%%%%%%%%%%%%%%%%%%%%%

\theoremstyle{definition}
\newtheorem{maar}{Määritelmä}[section]
\newtheorem{lemma}[maar]{Lemma}
\newtheorem{seur}[maar]{Seuraus}
\newtheorem{lause}[maar]{Lause}
\newtheorem{esim}[maar]{Esimerkki}

% englanninkieliset mukaan mukavuussyistä
\newtheorem{definition}{Määritelmä}[section]
%\newtheorem{lemma}[maar]{Lemma}
\newtheorem{corollary}[maar]{Seuraus}
\newtheorem{theorem}[maar]{Lause}
\newtheorem{example}[maar]{Esimerkki}

\theoremstyle{remark}
\newtheorem{huom}[maar]{Huomautus}
\newtheorem*{huom*}{Huomautus}

% englanninkieliset mukaan mukavuussyistä
\newtheorem{remark}[maar]{Huomautus}

\makeatletter
\newenvironment{tod}[1][\todname]{\par
  \pushQED{\qed}%
  \normalfont \topsep6\p@\@plus6\p@\relax
  \trivlist
  \item[\hskip\labelsep
        \itshape
    #1\@addpunct{.}]\ignorespaces
}{%
  \popQED\endtrivlist\@endpefalse
}
\providecommand{\todname}{Todistus}
\def\textfontii{\the\textfont\tw@}
\def\AmSTeX{{\textfontii A\kern-.1667em%
  \lower.5ex\hbox{M}\kern-.125emS}-\TeX\spacefactor1000 }
\makeatother

% Yleisimmin käyttettäville komennoille voi määritellä lyhynnemerkintöjä
% esimerkiksi
\newcommand{\Q}{\mathbb{Q}}
\newcommand{\R}{\mathbb{R}}
\newcommand{\Z}{\mathbb{Z}}
\newcommand{\C}{\mathbb{C}}
\newcommand{\abs}[1]{\left\vert{#1}\right\vert} % Itseisarvo


%%%%%%%%%%%%%%%%%%%%%%%%%%%%%%%%%%%%%%%%%%%%%%%%%%%%%%%%%%%%%%%%%%%%%%%%%%%%%%%%
%
%   käyttäjän täytettävä alue alkaa
%
%%%%%%%%%%%%%%%%%%%%%%%%%%%%%%%%%%%%%%%%%%%%%%%%%%%%%%%%%%%%%%%%%%%%%%%%%%%%%%%%
% käyttäjän omat määrittelyt
\newcommand{\bx}{\mathbf{x}}

\title{Olio-ohjelmoinnin metodiikka demo 1}
\author{Petri Holopainen, Matias Lappalainen, Topi Koivunen ja Oskari Vahala\\[1cm]
 Turun yliopisto}
\date{Syyskuu 2018}

\begin{document}    % tekstin muodostus alkaa
\maketitle          % tuottaa nimiön
%\tableofcontents    % tuottaa sisällysluettelon
%\clearpage % poista kommentti merkki, jos haluat sivunvaihdon sisällysluettelon jälkeen


\section{}

Ei ole luokkaa, jolla olisi toiminnallisuutta värien lisäykseen, eli päivitysten tekeminen olisi hankalaa. Värejä on vain kymmenen, joten mainiin täytyy lisätä lisää värejä tarvittaessa. Sovelluskehitykseen ohjelma on todella huono ja puutteellinen. Värejä voi määritellä vain heksalla, ei rgba:lla tai muilla värinmääritysmenetelmillä.


\addcontentsline{toc}{section}{Kirjallisuutta} % lisätään sisällysluettelon tietoihin
\renewcommand{\refname}{Kirjallisuutta} % otsikon muutos

\newpage

\title{Teht 1}
\begin{lstlisting}
public class Colors {

    ArrayList<Color> Colors;

    public Colors() {
        String[] colors = {"#33a1ee", "#a958a5", "#cb3f68", "#2d406c", "#248e82", "#c481fb", "#28666e", "#f06261", "#ffffff", "#000000"};
        Colors = new ArrayList<Color>();


        for(String c : colors) {
            Colors.add(new Color(c, c));
        }
    }

    /**
     * Method will pick one pre defined colour from arrayList
     * @.pre (ArrayList =! null) &&  0<=tmp<10
     * @.post RESULT.length == 7
     */
    public String randomize(ArrayList<Color> Colors) {
        Random r = new Random();
        int tmp = r.nextInt(9);
        return Colors.get(tmp).getName();
    }


    public ArrayList<Color> getColors() {
        return Colors;
    }
}
\end{lstlisting}

\newpage

\begin{lstlisting}
class Color {

    // Define class parameters
    String name;
    String hex;

    public Color(String name, String hex) {
        this.hex = hex;
        this.name = name;
    }

    /**
     * Method compares two colors, and returns boolean,
     * if they are equal
     * @.pre != null
     * @.post RESULT === (comparable.length === 7 && comparable[0] === '#')
     */
    public boolean isEqual(Color comparable) {
        return this.hex.equals(comparable.getName());
    }

    // Return color name
    public String getName() {
        return this.name;
    }

    // Return color name
    public String getHex() {
        return this.hex;
    }

}

\end{lstlisting}

\newpage

\section{asd}
tehtävän ensimmäinen metodi palauttaa väri olion annetuista coordinaateista.

\begin{lstlisting}
    /**
    ** @.pre |x| < width/2 && |y| < height/2 && x != null && y != null
    ** @.post RESULT == COLOR
    */
    readCell(int x, int y)
\end{lstlisting}

WriteCell methodi lisää väri olion koordinaatteihin x, y

\begin{lstlisting}
    /**
    ** @.pre x != null && y != null && Color == Color
    ** @.post true
    */
    writeCell(int x, int y, Color)
\end{lstlisting}

Width ja height kertovat ruudukon dimensiot.

Ainakin readCell methodin määrittelyä tulee muuttaa siten, että methodi voi palauttaa myös undefinedia, sillä muistipaikalla ei ole mitään.


\end{document}